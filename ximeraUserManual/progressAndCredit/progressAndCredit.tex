\documentclass{ximera}
\title{Progress and credit}
\author{Jason Nowell}

\begin{document}
\begin{abstract}
    How Ximera assigns progress to students as they complete an
    assignment.
\end{abstract}
\maketitle

For each distinct URL assigned, including an entire \verb!xourse! file, Ximera
reports a number between $0$ and $1$ with $1$ representing ``complete.''

\section{Progress within a Ximera file}

The total progress of a given Ximera file is broken up evenly with top level
environment.
For example, let's say you have the following Ximera document:

\begin{verbatim}
\documentclass{ximera}
\title{An Example Document}
\author{Jane Doe}
\begin{document}
\begin{abstract}
An example to help understand progress.
\end{abstract}
\maketitle

\begin{theorem}
If $x$ is a real number with $x=1$, then $x+x = 2$.
\end{theorem}

\begin{question}
In order to apply the theorem, $x$ must be
(select all that apply):
\begin{selectAll}
    \choice{A variable.}
    \choice[correct]{A real number.}
    \choice{An arbitrary constant.}
    \choice[correct]{Equal to $1$.}
    \choice{Trick question, $x$ is a letter.}
    \end{selectAll}
    \begin{problem}
    What does the theorem conclude that $x+x$ equals?
    \[
    x + x = \answer{2}
    \]
    \end{problem}
\end{question}
 
\begin{remark}
We have the theorem given to us, and it has two parts,
the ``if'' statement that sets the necessary hypotheses 
to apply the theorem, and the ``then'' statement which 
tells us the result.
\end{remark}

\begin{problem}
Compute $1+2$.
\[
1+2=\answer{3}
\]
\end{problem}
\end{document}
\end{verbatim}

Above there are four top level theorem environments above: A \verb!theorem!,
\verb!question!, \verb!remark!, and \verb!problem!. Each of these is worth
$25\%$ of the total credit.  The breakdown of points is described below:

\begin{description}
    \item[Theorem environment] Worth 25\% contingent on completion.
        \begin{description} 
            \item[No answerables] Automatically flagged as complete.
        \end{description}
    \item[Question environment] Worth 25\% contingent on completion.
        \begin{description}
            \item[selectAll] Environment Worth 50\% of the Problem
                Environments Credit - contingent on completion
                \begin{itemize}
                    \item Getting a ``correct'' flag gives completion.
                \end{itemize}

            \item[Problem environment] Problem Environment Worth 50\% of the Problem
                Environments Credit - contingent on completion
                \begin{itemize}
                    \item Getting a ``correct'' flag on the
                          \verb|\answer| box gives 100\% of the nested problems
                          credit for completion.
                \end{itemize}
        \end{description}

    \item[Remark environment] Worth 25\% contingent on
        completion.
        \begin{description}
            \item[No answerables] Automatically flagged as complete.
        \end{description}

    \item[Problem environment] Worth 25\% contingent on completion.
        \begin{itemize}
            \item Getting a ``correct'' flag gives completion.
        \end{itemize}
\end{description}

In table form:
\begin{center}
    \begin{tabular}{lcc}
        Environment & Local Level Credit & (Effective) Total
        Page Credit                                          \\\hline
        Theorem 1   &                    & 25\%              \\
        Problem 1   &                    & 25\%              \\
        selectAll   & 50\%               & 12.5\%            \\
        Problem 1.1 & 50\%               & 12.5\%            \\
        Explanation &                    & 25\%              \\
        Question 1  &                    & 25\%              \\
        answerBox   & 100\%              & 25\%
    \end{tabular}
\end{center}

So, there are three things to notice here.

\begin{enumerate}
    \item Environments that don't have interactives (i.e. some kind of
          way for students to contribute or answer for grade validation)
          automatically
          give credit as soon as the page is loaded. This also means that a
          large portion
          of the tile's credit could (theoretically) be given just for loading
          the page,
          without doing anything.
          \begin{itemize}
              \item This can actually be a good thing - it's a bit of a
                    morale booster, and in practice students typically
                    \textit{really} strive for
                    perfect scores on these things since they have time to do
                    it and want every
                    fraction of a point they can get. So getting some credit
                    for loading the page
                    doesn't seem to result in less effort spent on the page -
                    and tends to make
                    students feel better about their progress without actually
                    changing the amount
                    of progress or effort they expend.
          \end{itemize}
    \item The only things that can give credit \textit{are}
          environments. This means, if you put some kind of interactive (like
          an answer
          box) just floating around in the middle of the page without it being
          in an
          environment, it won't actually count toward credit for the tile -
          indeed, it
          may not even render or resolve correctly.
    \item Arguably most importantly: since credit is split evenly
          \textbf{at the top-level environment count only}, problems with lots
          of nested
          problems, can end up with nested problems that are worth
          significantly less
          compared to a problem that has no nested problems. This can be
          reasonable or
          not depending on the problem design, but it is worth bearing in mind
          as you
          design the problems on your page.

\end{enumerate}

\section{Progress within a \texttt{xourse}}

Credit for an assignment is determined at the xourse level, where each
tile (regardless of type or content) is equally weighted toward the percentage
complete. So if you have 7 activities and 3 practice tiles, then each of those
tiles is worth 10\% of the credit for the overall assignment. This is good to
keep in mind as you decide the content of each, since you want to be aware of
how that impacts the weight of the content \textit{within} the tile - although
students rarely seem to put that level of thought into it themselves.

\end{document}